% 08.27.12/pomp13.tex
\documentclass[12pt,leqno]{article}

\usepackage{amsmath,amssymb}
\usepackage{latexsym}
\usepackage{graphicx}
\usepackage{bm}

\setlength{\topmargin}{-0.50cm}
%\setlength{\textwidth}{15cm}
%\setlength{\evensidemargin}{-0.15cm}
%\setlength{\oddsidemargin}{0.5cm}
\setlength{\textheight}{20.5cm}

\setlength{\textwidth}{16.0cm}


\newtheorem{thm}{Theorem}%[section]
\newtheorem{cor}[thm]{Corollary}
\newtheorem{conj}[thm]{Conjecture}
\newtheorem{lem}[thm]{Lemma}
\newtheorem{prop}[thm]{Proposition}
\newtheorem{ax}{Axiom}
\newtheorem{rem}{Remark}

\newenvironment{proof}{\begin{trivlist}
                       \item[]{\bf Proof.}
                       \hspace{0cm}}{\hfill $\Box$
                       \end{trivlist}}


\def\cond{\mathop{\rm cond}}
\def\dft{\mathop{\rm DFT}}
\def\diag{\mathop{\rm diag}}
\def\Re{\mathop{\rm Re}}
\def\R{\mathbb{R}}
\def\C{\mathbb{C}}

\begin{document}

\baselineskip=16pt

\title{The Pompeiu problem}

\author{ A. G. Ramm
}
%\\
%$\dag$Mathematics Department, Kansas State University,\\
%Manhattan, KS 66506-2602, USA}

\renewcommand{\thefootnote}{\fnsymbol{footnote}}
%\footnotetext[1]{Email:
\footnotetext[1]{ Email: ramm@math.ksu.edu}

\date{}
\maketitle
\begin{abstract}
\baselineskip=14pt
Let $f \in L_{loc}^1 (\R^n)\cap \mathcal{S}'$, where
$\mathcal{S}'$ is the Schwartz class of distributions,  and
$$\int_{\sigma (D)} f(x) dx = 0 \quad \forall \sigma \in G, \qquad (*)$$
where  $D\subset \R^n$ is a bounded domain, the closure $\bar{D}$ of
which is diffeomorphic to a closed ball. Then the complement of $\bar{D}$
is connected and path connected.
By $G$ the group of all rigid motions of $\R^n$ is denoted. This group
consists of all translations and rotations.

It is conjectured that if $f\neq 0$ and (*) holds, then $D$ is a ball.
This conjecture is discussed in detail.

\end{abstract}

\textbf{MSC:} 35J05, 31B20

\textbf{Key words:} Symmetry problems; Pompeiu problem.

\section{Introduction}
In this paper the problem known as the Pompeiu problem is formulated and
discussed. This problem originated in a paper \cite{P}, of 1929.
The problem in a modern formulation is still open.

Dimitrie  Pompeiu (1873-1954) was born in Romania, got his Ph.D in 1905 at
the Sorbonne, in Paris, under the direction of H.Poincare. He is known
mainly for the Pompeiu problem and for the Cauchy-Pompeiu formula in
complex analysis.

Let us formulate the Pompeiu problem as it is understood today.

Let $f \in L_{loc}^1 (\R^n)\cap \mathcal {S}'$, where
$\mathcal{S}'$ is the Schwartz class of distributions,  and
\begin{equation}
\label{e1}
\int_{\sigma (D)} f(x) dx = 0 \quad \forall \sigma \in G,
\end{equation}
where $G$  is the group of all rigid motions of $\R^n$, $G$  consists of
all translations and rotations, and
 $D\subset \R^n$ is a bounded domain, the closure $\bar{D}$ of
which is diffeomorphic to a closed ball. Under these assumptions the
complement of $\bar{D}$
in $\R^n$ is connected and path connected, see \cite{H}, the isotopy
extension theorem.

In \cite{P} the following question was raised by D.Pompeiu:

{\bf Does (1) imply that  $f = 0$?}

If yes, then we say that  $D$  has $P$-property (Pompeiu's
property), and write $D\in P$. Otherwise, we say
that $D$ fails to have $P$-property, and write  $D\in \overline{P}$.
Pompeiu claimed in 1929 that every plane bounded domain has $P-$property.
This claim turned to be false:
a counterexample was given 15 years later in \cite{C}. The counterexample
is a
domain $D$ which is a disc, or a ball in $\R^n$ for $n> 2$. If $D$ is a
ball, then there are $f\not\equiv 0$ for which equation (1) holds.
The set of all $f\not\equiv 0$, for which equation
(1) holds, was constructed in \cite{R363}. There are infinitely many
(a continuum)  such $f$.
A bibliography on the Pompeiu
problem ($P-$problem) can be found in \cite{Z} and
in \cite{R363}.

The current formulation of the $P-$problem is the following:

{\it Prove that if $D\in \overline{P}$
then $D$ is a ball.}

We use the word ball also in the case $n=2$, when this word means
disc, and discuss the $P-$problem in detail. This problem leads
to some problems of general mathematical interest: a symmetry problem
for partial differential equations and a problem in harmonic analysis.

Let us make the following standing assumptions:

{\it Assumptions A:

$A_1):$ $D$ is a bounded domain, the closure of which is diffeomorphic to
a
closed ball, the boundary $S$ of $D$ is a closed connected $C^1-$smooth
surface,

$A_2):$ $D$ fails to have $P-$property.
}



Our main conjecture is:

{\bf Conjecture 1.} {\it If Assumptions A hold, then $D$ is a ball.}

In Section 2 this Conjecture  is discussed. We prove that this Conjecture
is equivalent to a symmetry problem for a partial differential equation.
Several symmetry problems were studied by the method based on Theorem 2,
see \cite{R512}-\cite{R614}.

{\bf Conjecture 2.} {\it If problem \eqref{e2} (see below) has a solution,
then $D$ is a ball.}

It is also equivalent to the following

{\bf Conjecture 3.} {\it  If Assumption $A_1$ holds and the Fourier
transform of the characteristic function of $D$ has a spherical surface of
zeros, then $D$ is a ball.}

\section{Discussion of the Conjectures}

It is proved in \cite{W} that if Assumptions A hold, then the boundary
$S$ of $D$ is real-analytic. It is proved in Theorem 3, below,  that
if Assumptions A hold, then  the problem
\begin{equation}
\label{e2}
 (\nabla^2 + k^2)u = 1\quad \text{in}\quad D,\qquad u\big{|}_S = 0,
\quad u_N\big{|}_S=0, \quad k^2 = const>0,
\end{equation}
has a solution.
% (see \cite{BST}, \cite{R363} and \cite{R470},Chapter 11).
In \eqref{e2} $N=N_s$ is the outer unit normal to $S$
pointing out
of $D$, $s\in S$ is a point on $S$.
%To make this paper essentially self-contained, a derivation of  relation
%\eqref{e2} is given in Theorem 2.

The basic results, on which our discussion of Conjectures 1, 2 and 3 is
based, are Theorems 1, 2 and 3. Theorems 1 and 2 were proved originally in
\cite{R382}, \cite{R363},  and in
\cite{R470}, Chapter 11. A result, equivalent to Theorem 3, has been
proved originally in \cite{BST} by a considerably longer and more
complicated argument. Our proof is borrowed essentially from
\cite{R470}, Chapter 11.
In the paper \cite{K} the null-varieties of the Fourier transform of
the characteristic function of a bounded domain $D$  are studied.
The properties of these varieties and the geometrical properies
of $D$ are related, of course, but it is not clear in what way.
Conjecture 3, if proved, is an interesting example of such a relation.


To make
the presentation essentially self-contained,  proofs of all of the
basic results are included in this paper.

{\bf Theorem 1.}  {\it If Assumptions A hold, then
\begin{equation}
\label{e3}
[s,N]=u_N, \quad \forall s\in S,
\end{equation}
where $[s,N]$ is the cross product in $\R^3$, and $u$ is a
vector-function that solves the problem
\begin{equation}
\label{e4}
(\nabla^2+k^2)u=0 \quad in \quad D, \quad u|_S=0.
\end{equation}
}
If $n=2$, then $D$ is a plane domain, $S$ is a curve, diffeomorphic
to a circle, $u$ is a scalar solution to equation (4), and equation (3)
yields $s_1N_2-s_2N_1=u_N, \quad \forall s\in S,$
where $N_j$, $j=1,2,$ are Cartesian coordinates of the unit normal
$N$ to $S$.
Indeed, if $n=2$ then the cross product of two vectors
$[s_1e_1+s_2e_2, N_1e_1+N_2e_2]$ is calculated by the formula
$[s,N]=(s_1N_2-s_2N_1)e_3,$ where $e_3$ is a unit vector,
orthogonal to the plane domain $D$, and the triple $\{e_j\}_{j=1}^3$
is a standard orthonormal basis in $\R^3$.

Let us state the following characterization of spheres.

{\bf Theorem 2.} {\it If $S$ is a smooth surface homeomorphic to a sphere
and $[s,N]=0$ on $S$, then $S$ is a sphere.}

The proof of Theorem 2 will be given in the coordinate system in which
the condition $[s,N]=0$ on $S$ is valid. This condition is not invariant
with respect to translations, although the geometrical conclusion, namely,
that $S$ is a sphere, is invariant with respect to translations.

It follows from Theorem 2 that {\it the conclusion of Conjecture 1 will
be established if one proves that $[s,N]=0$ on $S$}.

Let us start by  proving Theorem 2, then Theorem 1 is proved, and,
finally, we prove Theorem 3.
It is assumed throughout that $n=3$, but most of our arguments can be used
for any $n\ge 2$.

{\it Proof of Theorem 2.} Let $n=3$ and
assume that $s(p,q)$ is the parametric equation of the surface $S$.
The normal $N$ to $S$ is a vector directed along the vector
$[s_{p},s_{q}]$,
where $s_{p}$ is partial derivative with respect to the
parameter $p$.  The assumption $[s, N]=0$ on $S$, yields
\begin{equation}
\label{e5}
[s, [s_p, s_q]]=s_p s\cdot s_q-s_q s\cdot s_p=0.
\end{equation}
Since the vectors $s_p$ and $s_q$ are linearly independent
on a smooth surface, equation \eqref{e5}
implies
\begin{equation}
\label{e6}
\frac{\partial s^2}{\partial p}=0,\quad \frac{\partial s^2}
{\partial q}=0.
\end{equation}
Therefore
\begin{equation}
\label{e7}
s^2=const.
\end{equation}
This is an equation of a sphere in the coordinate system with the origin
at the center of the sphere.



Theorem 2 is proved.\hfill $\Box$

{\it Proof of Theorem 1.} Let $n=3$ and $\mathcal{N}$ denote the set of
all smooth solutions to \eqref{e4} in a ball $B$, containing $D$.
Multiply \eqref{e2} by
and arbitrary solution $h$ to equation \eqref{e4} in a ball $B$,
containing $D$, integrate by parts, take into account
the boundary conditions in \eqref{e2}, and get the relation
$$\int_D h(x)dx=0 \qquad \forall h\in \mathcal{N}.$$
Since $h\in \mathcal{N}$ implies $h(gx)\in \mathcal{N}$ $\forall g$,
where $g$ is an arbitrary rotation in $\R^3$ about the origin $O$,
one obtains
\begin{equation}
\label{e8}
\int_D h(gx)dx=0 \qquad \forall h\in \mathcal{N},\,\, \forall g.
\end{equation}
Let $O\in D$ be arbitrary,  and take an arbitrary straight
line $\ell$ passing through $O$ and directed along a
unit vector $\alpha$. Let
$g=g(\phi)$ be the rotation about
$\ell$ by an angle $\phi$ counterclockwise. Differentiate
\eqref{e8} with respect to $\phi$ and then set $\phi=0$, see \cite{R382},
cf \cite{R512}. The result is
\begin{equation}
\label{e9}
\int_D\nabla h(x)\cdot[\alpha, x]dx=0,
\end{equation}
where $[\alpha, x]$ is the cross product and $\cdot$ stands for the inner
product in $\R^3$. Equation  \eqref{e9} is invariant with respect to
translations because $\int_D\nabla h(x)dx=0$ $\forall h\in  \mathcal{N}$.
Indeed,  $h\in  \mathcal{N}$ implies $\nabla h \in  \mathcal{N}$.
Using the divergence theorem, the relation $\nabla \cdot [\alpha, x]=0$,
valid for any constant vector $\alpha$, and the arbitrariness of $\alpha$,
one derives from  \eqref{e9} the following relation
\begin{equation}
\label{e10}
\int_S h(s) [s, N]ds=0,\quad \forall h\in \mathcal{N},
\end{equation}
which is also invariant with respect to translations.
Indeed, $N$ is invariant under translations beacause $[s_p, s_q]$ is,
and $\int_S h(s) [a, N]ds=0$ for any constant vector $a$ because
$\int_S h(s) Nds=\int_D \nabla hdx=0$, as was pointed out above.
Let us derive from \eqref{e10} equation \eqref{e3}.
We need the following result.

{\bf Lemma 1.}{\it  The orthogonal complement of the set $M$ of the
restrictions of all $h\in \mathcal{N}$ to $S$ is a finite-dimensional
space spanned by the functions $u_{jN}$, where $\{u_j\}_{j=1}^J$
is the basis of the eigenspace of the Dirichlet Laplacian in
$D$, corresponding to the eigenvalue $k^2$.}

{\bf Remark 1}. It follows from \eqref{e10} that $[s,N]$ is orthogonal
in $L^2(S)$ to the set $M$. Therefore, by Proposition 1,  each of
the three components of $[s,N]$ must be  linear combinations of the
functions $u_{jN},\,\,1\le j\le J$. In other words, equation
\eqref{e3} holds.

{\it Proof of Lemma 1.} Note that the  result of Lemma 1 is
equivalent to the assertion that the boundary value problem
\begin{equation}
\label{e11}
(\nabla^2+k^2)h=0\quad in \quad D, \quad h|_S=f
\end{equation}
is solvable if and only if
\begin{equation}
\label{e12}
 \int_Sfu_{jN}ds=0, \quad 1\le j\le J.
\end{equation}
where $u_j$, $1\le j\le J,$ is a basis of the solutions to problem
\eqref{e4}.
The {\it necessity}  of conditions \eqref{e12} is proved by the relation
$$0=\int_Du_j(\nabla^2+k^2)hdx=-\int_Sfu_{jN}ds,$$
where an integration by parts  and the boundary condition $u_j=0$ on
$S$ were used, and equation \eqref{e4} for $h$
was taken into account.

The {\it sufficiency} of conditions \eqref{e12} is proved as follows.
Denote by $H^{m}(D)$ the usual Sobolev spaces. Given
an $f\in H^{3/2}(S)$, construct
an arbitrary $F\in H^2(D)$, such that $F|_S=f|_S$, and define
$h:=w+F$, where
\begin{equation}
\label{e13}
(\nabla^2+k^2)w=-(\nabla^2+k^2)F \quad in \quad D, \quad w|_S=0.
\end{equation}
If such $w$ exists, then $h=w+F$ solves problem \eqref{e11}.
For the existence of $w$ it is necessary and sufficient that
$$\int_D(\nabla^2+k^2)Fu_jdx=0,\quad 1\le j\le J.$$
An integration by parts shows that
these conditions are equivalent to conditions \eqref{e12} because
$u_j$ solve problem \eqref{e4}. Thus, Lemma 1 is
proved. \hfill $\Box$

Equation \eqref{e10} says that $[s,N]$ is orthogonal to the
set $M$, that is, to the restrictions of all $h\in \mathcal{N}$ to $S$.

{\bf Lemma  2}. {\it The set $M$ is dense in $L^2(S)$ in the set
of
all $\psi\in H^{3/2}(S)$ for which the boundary problem \eqref{e13} is
solvable.}

Therefore equation \eqref{e10} and
Lemma 1 imply  \eqref{e3}.

{\it Proof of  Lemma 2.}
Assume the contrary. Then for some $f\in H^{3/2}(S)$, $f\neq 0$,  problem
\eqref{e11} is solvable and
$$\eta(y):=\int_S f(s)\psi(s,y)ds=0\quad \forall y\in D':=\R^3\setminus
D,$$
where $\psi(x,y):=\frac{e^{ik|x-y|}}{4\pi |x-y|}\in \mathcal{N}$ for $y\in
B'$, that is, outside a ball containing $D$.  The function  $\eta$ is a
simple-layer potential which vanishes in $B'$, and by the unique
continuation property for solutions of the homogeneous helmholtz
equation, $\eta=0$ everywhere in $D'$. Thus, it vanishes on $S$.
Therefore, $\eta$ solves
problem \eqref{e4}. By the jump relation for the normal derivative of
$\eta$ across $S$, one has $f=\eta_{N}$, where $\eta_{N}$ is the limiting
value of the normal derivative of $\eta$ on $S$ from inside $D$. If
problem \eqref{e10} is solvable, then, as we have proved,
$f$ is orthogonal to all functions $u_{jN}$. The function $\eta_{N}$
is a linear combination of these functions. This and the relation
$f=\eta_{N}$ prove that $f=0$.
Consequently, we have proved the claimed density of $M$ in the set
of all $H^{3/2}(S)-$functions $f$ for which problem \eqref{e11} is
solvable.  Lemma 2 is proved. \hfill $\Box$

This completes the proof of Theorem 2. \hfill $\Box$


%Equation \eqref{e14} is not invariant with respect to translations.

{\bf Theorem 3}. {\it If (1) holds, then (2) holds.}

{\it Proof of Theorem 3.} Write \eqref{e1} as
$$\int_{\R^3}f(gx+y)\chi (x)dx=0\qquad \forall y\in \R^3,\quad \forall
g\in G,$$ where $\chi (x)$ is the characteristic function of $D$.
Applying the Fourier transform and the convolution theorem one gets
\begin{equation}
\label{eX}
\tilde{f}(\xi)\overline{\tilde{\chi}(g^{-1}\xi)}=0,
\end{equation}
 where
$\tilde{f}$ and $\tilde{\chi}$ are the Fourier transforms of $f$ and
$\chi$, respectively, and the overbar stands for the complex
conjugate. The Fourier transform of $f$ is understood in the sense
of distributions. The Fourier transform $\tilde{\chi}$ is an entire
function of exponential type because function $\chi$ has support
$\overline{D}$, which is a bounded set. Moreover, $\tilde{\chi}$ is
a uniformly bounded function of $\xi\in \R^n$. Therefore, the
product of the tempered distribution $\tilde{f}$ and the function
$\tilde{\chi}$ is a tempered distribution also.

 Since $g^{-1}$
runs through all the rotations, one can replace $g^{-1}$ by $g$. It
follows from \eqref{eX} that
\begin{equation}
\label{eY}
supp\, \tilde{f}=\cup_{k} C_k,\,\, \text {where} \,\, C_k:=\{\xi:
\tilde{\chi}(\xi)=0\,\,\, \forall \xi: \xi^2-k^2=0\}.
\end{equation}
 In other
words, the support of the distribution $\tilde{f}$ is a subset of
the union of spherical surfaces of zeros of $\tilde{\chi}$, the
Fourier transform of the characteristic function of the bounded
domain $D$.
 Since $\tilde{\chi}(\xi)$ is an entire
function of exponential type, vanishing on an irreducible algebraic
variety $\xi^2-k^2=0$ in $\C^3$, one concludes, using the division
lemma, that
\begin{equation}
\label{eZ}
\tilde{\chi}(\xi)=(\xi^2-k^2)\tilde{u}(\xi),
\end{equation}
 where $\tilde{u}$ is
an entire function of the same exponential type as $\tilde{\chi}$ (
see \cite{F}). Therefore, by the Paley-Wiener theorem, the
corresponding $u$ has compact support. Taking the inverse Fourier
transform of equation \eqref{eZ}, one gets:
\begin{equation}
\label{eW}
(-\nabla^2-k^2)u(x)=\chi(x)   \qquad in\quad \R^3, \quad u=0\, \text{if}\,\,
\quad |x|>R,
\end{equation}
 where $R>0$ is sufficiently large. By the elliptic
regularity results,  one concludes that  $u\in H^2_{loc}(\R^3)$.
Since $u$ solves the Helmholtz elliptic equation and vanishes near
infinity, that is, in the region  $|x|>R,$ the uniqueness of the
solution to the Cauchy problem to the equation \eqref{eW} and  the
path connectedness of the complement $D_1:=\bar{D}'$ of the closure
$\bar{D}$ of $D$ allow one to conclude that $u=0$ in $D_1$. The
connectedness and path connectedness of $D_1$ follow from our
Assumptions A and from the isotopy extension theorem (\cite{H}). If
$u=0$ in $D_1$ and $u\in H^2_{loc}(\R^3)$, it follows from the
Sobolev embedding theorem that the boundary conditions (2) hold.
Since $\chi(x)=1$ in $D$, equation (2) holds. Theorem 3 is proved.
\hfill $\Box$.

\section{Relation to analyticity}
The classical Morera theorem in complex analysis says
that if $\int_Cf(z)dz=0$ for any closed polygon $C$ in a domain $D$
of the complex plane, and if $f$ is continuously differentiable in $D$,
then $f$ is analytic in $D$. A simple proof is based on a version of
Green's formula: $0=\int_Cf(z)dz= 2i \int_{\Delta}\bar{\partial}f dxdy$.
Here $\Delta$ is the plane domain with the boundary $C$ and
$\bar{\partial}f:=\frac{f_x+if_y}{2}$. If $\int_{\Delta}\bar{\partial}f
dxdy=0$ for any polygon $\Delta$, then one easily derives that
$\bar{\partial}f=0$ evrywhere in $D$. This implies that $f$ is analytic in
$D$.

One may ask if the assumption that $f$ is continuously differentiable can
be replaced by a weaker assumption, and if the set of polygons can be
replaced by some other sets. The answer to the first question is easy:
if $f\in L^1_{loc}(D)$, then one considers a mollified function
$f_\epsilon(z):=\int_{\zeta: |z-\zeta|\le
\epsilon}\omega_\epsilon(z-\zeta)f(\zeta)dudv$, where $\zeta=u+iv$ and
$\omega_\epsilon(z)$ is the standard mollifying kernel (cite{Ho}, p.14).
It is known that $f_\epsilon\to f$ in $L^1(D)$ as $\epsilon \to 0$, and
one can select a subsequence $\epsilon_j\to 0$, such that
$f_{\epsilon_j}\to f$ almost everywhere in $D$. If
$\int_Cf(z)dz=0$ for any closed polygon $C$, then
$\int_Cf_\epsilon(z)dz=0$ for any closed polygon $C$, and the above
argument for a $C^1-$smooth $f$ leads to the conclusion that
$f_\epsilon$ is analytic in $D$ for all sufficiently small $\epsilon$.
Since a sequence $f_{\epsilon_j}$ of analytic functions
converges to $f$ in $L^1(D)$ and almost everywhere in $D$,
one concludes that $f$ is analytic in $D$. The second question:
can one replace the set of polygons by other sets is less simple.
For example,  one cannot replace polygons by the set $\sigma(B)$,
where $B$ is a ball. Indeed, using the above argument one arrives
at the relation $\int_{\sigma(B)}\bar{\partial}fdxdy=0$, and this does not
imply that $\bar{\partial}f=0$.














\newpage

\begin{thebibliography}{99}

\baselineskip=16pt
\bibitem{BST} L. Brown, B. Schreiber, B. Taylor,   Spectral
synthesis and the Pompeiu problem, {\it Ann. Inst. Fourier}  {\bf
23} no. 3 (1973) 125-154.

\bibitem{C} L. Chakalov, Sur un probleme de D.Pompeiu, {\it Godishnik
Univ. Sofia, Fac. Phys-Math.} {\bf 40} (1944) 1-14.


%\bibitem{CH} T. Chatelain and A. Henrot, Some results about Schiffer's
%conjectures,  Inverse Problems,  15, (1999), 647-658.

\bibitem{F} B. Fuks, {\it Theory of analytic functions of several
variables.} AMS, Providence RI, 1963.

\bibitem{H} M. Hirsch, {\it Differential topology.} Springer-Verlag, New
York, 1976.

\bibitem{Ho} L.H\"ormander, {The analysis of linear partial differential
operators I.} Springer-Verlag, Berlin, 1983.

\bibitem{K} T. Kobayashi, Asymptotic behavior of the null variety for a
convex domain in a non-positively curved space form,
{\it J. Fac. Sci.Univ. Tokyo Sect IA Math.}{\bf 36} (1989) 389-478.

\bibitem{P}  D. Pompeiu,   Sur une propriete
integrale des fonctions de deux variables reelles,  {\it Bull. Sci.
Acad. Roy. Belgique} {\bf 5} no 15 (1929) 265-269.

\bibitem{R363} A. G. Ramm,  The Pompeiu problem, {\it Applicable
Analysis} {\bf 64} no 1-2 (1997) 19-26.

\bibitem{R382} A. G. Ramm,
 Necessary and sufficient condition for a domain, which fails to have
Pompeiu property, to be a ball, {\it Journ. of Inverse and Ill-Posed
Probl.} {\bf 6} no 2, (1998), 165-171.

\bibitem{R470} A. G. Ramm, {\it Inverse Problems.} Springer, New York,
2005.

\bibitem{R512} A. G. Ramm, A symmetry problem, {\it Ann. Polon. Math.} {\bf
92} (2007) 49-54.

\bibitem{R556} A. G. Ramm, Symmetry problems 2, {\it Annal. Polon. Math.} {\bf
96} no 1 (2009) 61-64.

\bibitem{R614} A. G. Ramm, Symmetry problem, {\it Proc. Amer. Math. Soc.}
(forthcoming)

\bibitem{W} S. Williams,
Analyticity of the boundary for Lipschitz domains without Pompeiu
property, {\it Indiana Univ.\ Math.\ Journ.} {\bf 30} (1981) 357-369.

\bibitem{Z} L. Zalcman, A bibliographical
survey of the Pompeiu Problem, in the book {\it Approximation by
solutions of partial differential equations.} Edited by B.Fuglede.
Kluwer Acad., Dordrecht, 1992, pp. 177-186.

\end{thebibliography}








Department of Mathematics, Kansas State University, Manhattan, KS
66506-2602,


email:     ramm@math.ksu.edu


\end{document}
