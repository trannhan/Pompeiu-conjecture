%663.tex
%\documentclass{amsart}
\documentclass[12pt]{article}
\usepackage{amsxtra,amssymb,amsthm,amsmath,latexsym}

\textheight=8.5truein \voffset=-.5truein
\textwidth=6.25truein \hoffset=-.5truein
%\renewcommand{\baselinestretch}{1.5}

\renewcommand{\thefootnote}{{~}}

\theoremstyle{plain}
\newtheorem{definition}{Definition}[section]
\newtheorem{theorem}{Theorem}[section]
%\theorembodyfont{\rmfamily}
\newtheorem{proposition}{Proposition}[section]
\newtheorem*{claim}{Claim}
\newtheorem{lemma}{Lemma}[section]
\newtheorem{corollary}{Corollary}[section]
\newtheorem{remark}{Remark}[section]
\numberwithin{equation}{section}

\newcommand{\res}{\operatorname{Res}}

\newcommand{\refT}[1]{Theorem~\ref{T:#1}}
\newcommand{\refS}[1]{Section~\ref{S:#1}}
\newcommand{\refU}[1]{Subsection~\ref{U:#1}}
\newcommand{\refL}[1]{Lemma~\ref{L:#1}}
\newcommand{\refC}[1]{Chapter~\ref{C:#1}}
\newcommand{\refCL}[1]{Claim~\ref{CL:#1}}
\newcommand{\refP}[1]{Proposition~\ref{P:#1}}
\newcommand{\refD}[1]{Definition~\ref{D:#1}}
\newcommand{\refR}[1]{Remark~\ref{R:#1}}

\def\ds{\displaystyle}
\def\nd{\noindent}
\def\bysame{\rule{.5in}{.005in},\ }
\def\qed{{\hfill $\Box$}}
\def\ra{\rightarrow}
\def\lra{\longrightarrow}
\def\R{{\mathbb R}}
\def\N{{\mathbb N}}
\def\Z{{\mathbb Z}}
\def\C{{\mathbb C}}
\def\calD{{\mathcal D}}
\def\doteta{\dot{\eta}}
\def\dotf{\dot{f}}
\def\dotg{\dot{g}}
\def\doth{\dot{h}}
\def\dotu{\dot{u}}
\def\tildeR{\widetilde R}
\def\oH{{\overset{\circ}{H}}}
\def\oH1{{\overset{\circ}{H}\kern-.02in{}^1}}
\def\l{\ell}
\def\cop{\bot\hskip-.075in\bot}
\def\ind{{\hbox{\,ind\,}}}
\def\Im{{\hbox{\,Im\,}}}
\def\supp{{\hbox{\,supp\,}}}
\def\const{{\hbox{\,const\,}}}
\def\loc{{\hbox{\,loc\,}}}
\def\bee{\begin{equation*}}
\def\eee{\end{equation*}}
\def\be{\begin{equation}}
\def\ee{\end{equation}}


%\renewcommand{\theorem}{\arabic{theorem}}%\setcounter{theorem}{1}

\begin{document}

%\begin{titlepage}
\title{Conjecture}

\author{Alexander G. Ramm \\
Mathematics Department, Kansas State University\\
Manhattan, KS 66506, USA\\
email:     ramm@math.ksu.edu\\
http://www.math.ksu.edu/\,$\widetilde{\ }$\,ramm}

\date{}

\maketitle\thispagestyle{empty}

\begin{abstract}

\end{abstract}


 Math subject classification: 47A05

Key words: closed graph theorem; closed linear operator; uniform boundedness principle; new short proof of the closed graph theorem.


\section{INTRODUCTION.}\label{S:1}

I'd like to formulate a research question that so far I did not answer theoretically.
Can this question be answered using a computer program?

Let M be a positive integer, $f_m=\overline{f_{-m}}, f_m$ are numbers, $-M \le m \le M, m$ are integers, $\overline{.}$ stands for complex conjugate,
\be 
 f(\theta):= \sum_{m=-M}^M f_m e^{im\theta}, \quad 0<c_1\le f(\theta)\le c_2,  
\ee
$c_1, c_2$ are constants.

Let $j_m$ be integers, such that
\be \label{eq2}
	0\le j_m\le n+2;     \sum_{m=-M}^M j_m=n+2;     \sum_{m=-M}^M mj_m=-n,       
\ee
where
\be
	C^{n+2}_{...j_m...}:=\frac {(n+2)!}{(j_{-M})!.....(j_m)!...(j_M)!}
\ee
is a multinomial coefficient:  
\be
	(\sum_{m=-M}^M a_m)^n=\sum_{j_m}C^{n}_{...j_m...}\prod_{m=-M}^M a_m^{j_m}
\ee	
Assume:

1. \eqref{eq2}
and

2.  $\sum_{j_m}C^{n+2}_{...j_m...}\prod_{m=-M}^M f_m^{j_m}=0,$
where 1. and 2. hold for all $n>n_0$, and $n_0$ is an arbitrary large fixed positive integer.
Here $M$ does not depend on $n$.

My Conjecture:  Assumptions 1. and 2. imply that $f_m=0 \quad\forall  m\neq 0$.

I have proved this conjecture  for M=1.

Can one design a computer program to prove this conjecture for any (finite) integer $M$?
One may start with M=2.
If this conjecture is proved, then an important open (since 1929) problem will
be solved.


\begin{thebibliography}{999}

\bibitem{DS} N.Dunford, J. Schwartz,  Linear operators,  Part I, Interscience, New York, 1958.

\bibitem{H}  P. Halmos, A Hilbert space problem book, Springer-Verlag, New York, 1974. (problems 52 and 58)

\bibitem{He} J. Hennefeld, A non-topological proof of the uniform boundedness theorem, Amer. Math. Monthly,
87, (1980), 217.

\bibitem{Ho} S. Holland,  A Hilbert space proof of the Banach-Steinhaus theorem, Amer. Math. Monthly,
76, (1969), 40-41.

\bibitem{K} T. Kato,  Perturbation theory for linear operators, Springer-Verlag, New York, 1984.

\bibitem{S} A. Sokal,  A relally simple elementary proof of the uniform boundedness theorem,
Amer. Math. Monthly, 118, (2011), 450-452.

\bibitem{Y} K. Yosida,  Functional analysis, Springer, New York, 1980.

\end{thebibliography}


\end{document}
